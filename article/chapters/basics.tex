\section{Basics}
\subsection{LLVM Project}

The LLVM Project~\cite{LLVM:CGO04} is a collection of components which can be used to build compilers and related tools
for various programming languages and target platforms.
In this report we will have a look at how the LLVM code generation process works for most target architectures
and how LLVM can optimize code to take advantage of SIMD intruction sets.

LLVM began as a research project at the University of Illinois, with the goal of building a fast and modular compiler.
The idea is to have a set of modular compiler components to reduce the time and cost of building compilers
particular use cases. By having components compilers can share them and profit from advances in every component.
The LLVM includes (among others) the following technologies:
\begin{itemize}
    \item A number of compiler frontends
    \item Various optimizers
    \item Code Generation Framework
    \item An advanced debugger (LLDB)
    \item An symbolic execution engine (KLEE)
\end{itemize}

\subsection{LLVM Intermediate Representation}
\label{subsec:llvm_ir}

Every compiler frontend of LLVM translates the source code into LLVM's intermediate representation (IR).
The LLVM IR is essentially an high level assembly language, which can be used as basis for optimization,
vectorization, code generation, static analysis, symbolic execution, and much more.
The IR is used to represent the program logic in memory as well as on the disk. All compiler transformations
take place on the IR or start with the IR. The advantage of the LLVM IR compared to bitcode of other platforms
(e.g the Java VM's bitcode) is that it is human readable. For example the code in~\ref{lst:llvm_ir} implements
the function $f(aa, bb, cc) = (aa + bb) / cc$ fo signed 32 bit integers. The LLVM IR code is still easily
readable for someone who is at least familiar with C and some assembly code. 
\label{lst:llvm_ir}
\begin{lstlisting}[language=LLVM,gobble=4]
define i32 @foo(i32 %aa, i32 %bb, i32 %cc) {
    entry:
    %add = add nsw i32 %aa, %bb
    %div = sdiv i32 %add, %cc
    ret i32 %div
}
\end{lstlisting}

The IR was designed to be light-weight and low-level, while style being very expressive (as opposed to actual assembly code).
It's possible to map high level language constructs cleanly into the IR code, but still having type information available.
This is important for optimization and code generation. Every code generator can use the type information to cleanly transform
the IR into valid machine code, optimizers can use the IR to perform target independent optimizations.
Additionally the IR natively supports vector types. We will discuss the topics of vectorization~\ref{sec:autovec} 
and code generation~\ref{sec:codegen} in the next sections.